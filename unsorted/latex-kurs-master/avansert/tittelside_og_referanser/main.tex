\documentclass[12pt,a4paper, norsk]{article}
\usepackage[utf8]{inputenc}
\usepackage{graphicx}
\usepackage{parskip}
\usepackage[margin=2.5cm]{geometry}
\usepackage[colorlinks=true, a4paper=true, pdfstartview=FitV,
linkcolor=blue, citecolor=blue, urlcolor=blue]{hyperref}

\usepackage[norsk]{babel}
\usepackage[threshold=2]{csquotes}
\usepackage[style=apa,sortcites,backend=biber, natbib, hyperref=true]{biblatex}
\DeclareLanguageMapping{norwegian}{norwegian-apa}
\addbibresource{references.bib}

\begin{document}


\begin{titlepage}
	\centering
	\vspace{2.5cm}
	\includegraphics[width=0.15\textwidth]{figs/logo}\par\vspace{1cm}
	{\scshape\LARGE Norges Smart i Hodet Universitet \par}
	\vspace{1cm}
	{\scshape\Large Mitt Teite Fag (LØK1337)\par}
	\vspace{1.5cm}
	{\huge\bfseries Denne tittelen viser at jeg har gått på skole i flere år\par}
	\vspace{2cm}
	{\Large Andreas Drivenes\par}
	\vfill
	{\large \today\par}
\end{titlepage}

Hver eneste dag gir mennesker i fra seg digital informasjon som lagres
rundt omkring i verden. Facebook-meldinger, Twitter-statuser,
bankopplysninger, bilder og dokumenter gir et ganske komplett bilde
av hva du som person er opptatt av. Dette er verdifulle
data som kan brukes til markedsføring og til å skape bedre tjenester.
Skyggesiden er at digital informasjon, som all annen informasjon, er sårbar for utnyttelse
av aktører med onde hensikter, og sårbar på en helt annen måte enn tidligere.

Løsningen «';--have i been pwned?» \footnote{https://haveibeenpwned.com/}
har 1 994 449 415 kontoer fra 170
kompromitterte nettsteder. Her kan brukere sjekke om de har fått kontoen
sin med i et datainnbrudd ved å oppgi e-postadresse. Hvordan harmonerer disse
tallene med at selskaper tar informasjonssikkerhet på alvor? Har digitale
tjenester egentlig incentiver som virker for å passe på personvernet til brukerne?

\citet{datatilsynet} skriver at
«innebygd personvern betyr at det tas hensyn til personvernet i alle utviklingsfaser
av et system eller en løsning. Det er både kostnadsbesparende og mer effektivt enn
å endre et ferdig system.» Innebygget personvern
er tatt med i den nye personvernforordningen til EU.
Artikkel 25 sier at standarden til et system skal være at man kun prosesserer og lagrer de
personopplysningene som skal til for å løse de oppgavene systemet har \citep{eu-regulation}.

Når det svir mer for lommeboken å ha dårlig behandling av personlige data gir det økonomiske
incentiver til å bake inn personvern i det løsningen lages, selv for nyoppstartede
selskaper. Eksplisitt samtykke for brukeren gir selskapene en mulighet til å
begrense hvor mye data man faktisk trenger for å drive tjenesten sin.

Hva skal man så gjøre om man har lyst til å forsvinne fra nettets spindelvev av personlig
informasjon? I likhet med Datatilsynets \url{slettmeg.no} gir personvernforordningen deg
rett til å be om å bli fullstendig slettet fra tjenesten.



\printbibliography[heading=bibintoc, title={Litteraturliste}]

\end{document}
