\documentclass[a4paper]{article}

\usepackage[utf8]{inputenc}
\usepackage{graphicx}
\usepackage{blindtext}
\usepackage{amsmath}

% comment

\title{Krasj-kurs i \LaTeX}
\author{Jon Ryfetten }
\date{\today}

\begin{document}

\maketitle

\begin{abstract}
    Hei, dete er et sammendrag av hele foredraget
    \blindtext
\end{abstract}

\section{The basics}
    \begin{enumerate}
        \item Første
        \item Andre
        \item Tredje
        \begin{itemize}
            \item Underpunkt
            
        \end{itemize}
    \end{enumerate}
    
\subsection{Bilder}

\begin{figure}[h]
    \centering
    \includegraphics[width=0.2\linewidth]{icon}
    \caption{Caption}
    \label{fig:icon}
\end{figure}


\subsection{Tabell}

    \begin{table}
        \centering      % l = første kollone left, center, right
        \begin{tabular} {| l | c | r |} \hline
            1 & 2 & 3 \\ \hline
            4 & 5 & 6 \\ \hline
            7 & 8 & 9 \\ \hline
        \end{tabular}
    \end{table}

\subsection{Matte}

Kvadratroten er definert som $y = \sqrt{x}$. Dette kan vi skrive inne i teksten.

\begin{equation*} \label{eq:combinatorics}
    \frac{n!}{k!(n-k)!} = \binom{n}{k}
\end{equation*}


\begin{equation}
  
    u(x) = 
        \begin{cases}
            \exp{x} & \text{if } x \geq 0 \\
            1 & \text{if } x < 0
        \end{cases}
    
\end{equation}

I dette prosjektet fant vi en fig som du kan se i figur \ref{fig:icon}


% \bibliography{eferences}
% \bibliographystyle{unsrt}




\end{document}
